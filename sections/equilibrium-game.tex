\section{Traffic User Equilibrium as a Differential Game}

In this section we lay the groundwork for reasoning about traffic equilibrium. We can rephrase \eqref{eqn:ui-vi} in first order logic in \eqref{eqn:ui-vi:fol}. For brevity let $\Omega = \Omega_d \cap \Omega_s$ denote the feasible set of traffic flows.

\begin{align}
    \exists \mathbf{y}\in \Omega \forall \mathbf{z}\in \Omega (\mathbf{t}(\mathbf{y})(\mathbf{z}- \mathbf{y}) &\geq 0) \label{eqn:ui-vi:fol}
\end{align}

If we can express $\mathbf{t}(\mathbf{y})(\mathbf{z}- \mathbf{y})$ as the prime of some term $F$ then we leverage the differential game invariant rule (DGI) introduced in \citet{DBLP:journals/tocl/Platzer17}. Here we present a special case of DGI.

\begin{prooftree}
    \AxiomC{$\exists y\in Y \forall z\in Z [x'\assigned f(x,y,z)](F' \geq 0)$}
    \LeftLabel{DGI}
    \UnaryInfC{$F\geq 0 \implies [x'=f(x,y,z) \& y \in Y^d \& z\in Z] (F\geq 0)$}
\end{prooftree}

In this setting, we can interpret traffic equilibrium as a game where Demon proposes a traffic flow $\mathbf{y}$. This flow induces a travel cost across each link in the network. Angel tries to propose an alternate traffic flow $\mathbf{z}$ which, using Demons link costs $\mathbf{t}(\mathbf{y})$, results in a lower total cost in the network.

What remains is to find a meaningful interpretation for $x$ (and by extension $x'$) which leads to $F' = \mathbf{t}(\mathbf{y})(\mathbf{z}- \mathbf{y})$.
