\section{Introduction}

The traffic network equilibrium problem, also known as the traffic assignment problem, aims to identify the traffic patterns on transportation networks which arise after a set of travellers determine the least cost path through the network between their origin and destination \citep{nagurney2009netecon}.
The travel cost along each link in the network is a function of the traffic volume on the link, so each traveller must take into account the actions of other travellers when determining their own least cost route.

Solving the traffic assignment problem is a critical to transportation planning because it can compute roadway usage given travel demand and some mild assumptions on traveller behavior.
Often, planning applications take the form of a transportation network design problem (NDP) in which design parameters are optimized with traffic equilibrium as a constraint.
The design parameters influence the cost function of the network.
As a result, different values of the design parameters induce different equilibrium traffic flow on the network.
For example, \citet{sheffi1983optimal} find the signal timings that minimize travel delay subject to the constraint that traffic is at equilibrium with respect to the cost induced by that signal timing.

This project identifies a dynamical system for a very simple network over both traffic flow and the design parameter that characterizes the equilibria as the value of the design parameters changes.
In particular this dynamical system 
\begin{enumerate}
    \item preserves as an invariant the equilibrium state as the design parameter smoothly changes, and 
    \item is expressible in differential dynamic logic, the logic we have used in this course.
\end{enumerate}

This formulation lays the foundation for a new approach to solving NDPs by characterizing their constraint set in a novel manner.\\

The remainder of this paper is organized as follows.
\Cref{sec:background} defines traffic equilibrium and offers a brief overview of existing formulations and applications of traffic equilibrium.
\Cref{sec:motivating-example} Briefly introduces the morning commute problem: the canonical motivating example for the \textit{static traffic assignment problem}.
\Cref{sec:methods} introduces the notation used to formalize network equilibrium.
\Cref{sec:equilibrium-dynamics} introduces two different dynamical systems formulations for network equilibrium.
\Cref{sec:single-player-control} formulates the NDP problem in terms of projected dynamical systems, one of the two dynamical systems introduced in the previous section.
\Cref{sec:single-player-example} introduces a concrete simple network and equilibrium-preserving dynamics. It is shown that these dynamics preserve equilibrium under a changing control parameter.
Finally \cref{sec:discussion} reviews what has been done in this project as well as the limitations of this approach.
Additionally, several avenues for further work are identified.

