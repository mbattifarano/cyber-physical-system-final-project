\section{Next Steps}

In this section I outline the next steps for the project.
One thing I am still thinking about from the white paper feedback is viewing the problem from the perspective of an inf/sup condition on differential game values.
I'll continue to think about this and hopefully work it in to one of the following more concrete tasks I have listed below.

\subsection{Further consideration of equilibrium dynamics}

This section will further elaborate on the route choice adjustment process \eqref{eqn:project-dynamical-system}.
The dynamical system established has limited practical use on its own: it simply describes an algorithm for (eventually) finding an equilibrium given any feasible link flow.
However, in the context of optimal control or a differential game it has a reasonable practical interpretation: that of a reaction to a change in link costs. 
In practice, a change in toll price or traffic signal timing has an immediate effect on travel cost but a delayed effect on traveller route choice.
In addition to the question of which control input is optimal (or simply good enough) a traffic controller might want to know if the traffic flow will pass through any undesirable states on its way to the new equilibrium.
This interpretation will be explored more fully.\\

With equilibrium dynamics established we may now introduce control inputs to the link cost function.

\subsection{Single player control}

Suppose a traffic controller is able to influence the link cost function, for example, by applying a toll to one or more links.
Here we are interested in examining the dynamical system governed by \eqref{eqn:single-player-dynamics}. 

\begin{align}
    \mathbf{x}' &= \Pi_{\Omega} (\mathbf{x}, -\mathbf{t}(\mathbf{y}, \mathbf{x}))\label{eqn:single-player-dynamics}
\end{align}

This is a trivial differential game in the sense that the dynamics do not additionally depend on a second players control.
It is typically analyzed in the setting of optimal control theory. \citep{bardi2008optimal}

In this section appropriate conditions on $\mathbf{t}$ and the control input $\mathbf{y}$ will be developed so that \crefrange{thm:correspondence}{thm:convergence} hold.
Additionally, the simple case in which the control input is constant will be examined as it can be readily expressed as a hybrid system.
Suitability of this formulation to applications will then be discussed.

\subsection{Two-player game}

As before, suppose a traffic controller is able to influence the link cost function.
Further suppose that there is an unknown disturbance in the network which has the effect of changing the link cost function, for example, a felled tree, or a malfunctioning traffic signal.
Here we are interested in examining the differential game given by \eqref{eqn:two-player-dynamics}.

\begin{align}
    \mathbf{x}' &= \Pi_{\Omega} (\mathbf{x}, -\mathbf{t}(\mathbf{y}, \mathbf{z}, \mathbf{x}))\label{eqn:two-player-dynamics}
\end{align}

This problem is an example of optimal control in the presence of an unknown disturbance, a well-established class of problems that are formulated as differential games. \citep{bardi2008optimal}
In this section, as in the previous, appropriate conditions on $\mathbf{t}$ and the control inputs $\mathbf{y}$ and $\mathbf{z}$ will be developed so that \crefrange{thm:correspondence}{thm:convergence} hold.
The upper and lower values of this game will be defined and examined.
Finally, the suitability of this formulation to applications will be discussed.