\section{Dynamical Systems Formulation of Traffic Equilibrium}\label{sec:equilibrium-dynamics}

In this section two related dynamical systems formulations of traffic equilibrium are introduced.

\subsection{Projected Dynamical Systems}

The variational inequality formulation of the traffic equilibrium condition is a static condition.
Of particular interest in re-framing traffic equilibrium in the logic of hybrid systems is to identify a dynamical system which not only preserves traffic equilibrium as an invariant but pushes a non-equilibrium state toward equilibrium.
We can develop a dynamical system from the decision process from the motivating example in \cref{sec:motivating-example}.
Let $\mathbf{x}$ denote yesterday's link flow.
When choosing today's routes, the drivers wish to pick a feasible link flow $\mathbf{\tilde{x}}$ which reduces travel costs.
One way to achieve this is to simply move $\mathbf{x}$ in the direction that reduces its link costs by some step size $\alpha \geq 0$, i.e. $\mathbf{x}-\alpha \mathbf{t}(\mathbf{x})$, and then project it back onto the feasible set to ensure that the resulting link flow remains feasible.

\citet{nagurney1997projected} introduce a \textit{Projected Dynamical System} formulation of traffic equilibrium given by \eqref{eqn:project-dynamical-system} which extends the discrete decision process to a continuous one.

\begin{align}
    \mathbf{x}' &= \Pi_{\Omega}(\mathbf{x}, -\mathbf{t}(\mathbf{x}))\label{eqn:project-dynamical-system}\\
    \intertext{with link flow $\mathbf{x} \in \Omega$ and,}
    \Pi_{\Omega}(x, v) &= \lim_{\varepsilon\to 0}\frac{P_{\Omega}(x + \varepsilon v) - P_{\Omega}(x)}{\varepsilon}\\
        &= \lim_{\varepsilon\to 0}\frac{P_{\Omega}(x + \varepsilon v) - x}{\varepsilon}\\
    P_{K}(u) &= \arg\min_{v\in K} \| u-v\|
\end{align}

The operator $\Pi_{\Omega}$ can be thought of as a G\^{a}teaux directional derivative along the negative of the link cost of the norm projection onto $\Omega$, $P_{\Omega}$ for $x\in \Omega$.
Under this model, link flows change in the feasible direction that offers the largest reduction in the users travel costs.
The intuition behind this operator is that traffic volume should move from higher cost links to lower cost links.
By moving in the direction of $-t(x)$ the traffic volume on high cost links is reduced faster than the volume on lower cost links.
The net effect of the projection back onto the feasible set is then to move volume from higher cost links to lower cost links until the costs are equalized.

It's important to note that solutions of \eqref{eqn:project-dynamical-system} are not functions of real time, but of some abstract decision space.
The main result presented in \citet{nagurney1997projected} is to show that one my discretize \eqref{eqn:project-dynamical-system} to produce an algorithm which converges on the true solution.
In this setting, we may think of this system as representing, somewhat counter-intuitively, the long term evolution of traffic flow.

\begin{figure}[!ht]
    \centering
    \includegraphics[width=0.4\textwidth]{figures/equilibrium-projection.png}
    %\begin{tikzpicture}
    \coordinate (origin) at (0, 0);
    
    \coordinate (vertical extent) at (0, 6);
    \coordinate (horizontal extent) at (6, 0);
    
    % axes
    \draw 
        (origin) -- (vertical extent)
        (origin) -- (horizontal extent)
    ;
    
    % supply-feasible region
    \draw (5, 0) arc (0:90:5cm) node[below right] {$\Omega_s$};
    % demand-feasible region
    \begin{scope}[shift={(3.5, 3.5)}]
        \draw [rotate=-45] (0, 0) ellipse (3cm and 2cm) node [below right=2cm] {$\Omega_d$}; 
    \end{scope}
    
    % cost lines
    \draw 
    
\end{tikzpicture}
    \caption{Visual depiction of the dynamical system in \eqref{eqn:project-dynamical-system}. Here, $D\cap S$ represents the feasible region $\Omega$, $u(\cdot) = -t(\cdot)$ and $Tf$ is the projection of $f+u(f)$ onto $\Omega$. The vector field represents $u$. The flow $h$ is an equilibrium. We see that the equilibrium flow $h$ is a fixed point of $T$ whereas the $T$ applied to some non-equilibrium point $f$, is not a fixed point ($Tf \neq f$) and nudges the system toward equilibrium. The operator $\Pi_{\Omega}$ can be considered the infinitesimal version of the operator $T$ in the diagram.}
    \label{fig:equilibrium-projection}
\end{figure}

As a succinct visual aid of this process, consider \cref{fig:equilibrium-projection} re-printed here from  \citet{smith1979existence}. 
Lets consider how the route choice adjustment process acts on an arbitrary feasible link flow $x$ in \cref{fig:equilibrium-projection}.
Suppose $x$ is in the interior of $D\cap S$.
In this case, $x'$ is simply $-t$ and the link flow will evolve toward the boundary.
Upon reaching the boundary, $x$ can no longer move in the direction of $-t$ and remain in the feasible set.
The dynamics ensure this through use of the projection operator.
As a result, $x$ evolves in the direction most closely aligned with $-t$ that still points in the direction of $\Omega$.
The link flow will then move along the boundary, approaching the equilibrium point $h$.  
At the equilibrium point we see that $\Pi_{\omega}(h, -t(h))=0$ since $P_{\Omega}(h- \varepsilon t(h)) = h$ for all $\varepsilon \geq 0$.\\

We repeat three important theorems proved in \citet{nagurney1997projected}.

\begin{theorem}
    \label{thm:correspondence}
    A link flow $x^*\in \Omega$ satisfies the variational inequality problem \eqref{eqn:ui-vi} if and only if it is a stationary point for the differential equation \eqref{eqn:project-dynamical-system}, that is,
    $$0= \Pi_{\Omega}(\mathbf{x}, -\mathbf{t}(\mathbf{x}))$$
\end{theorem}

\begin{definition}[Stability]
    \label{def:stability}
    The route choice adjustment process given by \eqref{eqn:project-dynamical-system} is \textbf{stable} if for every initial flow pattern $\mathbf{x}_0\in \Omega$ and every equilibrium flow pattern $\mathbf{x}^*$, the distance $\|\mathbf{x}^* - \mathbf{x}_0(s)\|$ is monotone non-increasing in $s$.
\end{definition}

\begin{theorem}
    The route choice adjustment process is \textit{stable} if the link cost $t$ is monotone increasing in the link flow.
\end{theorem}

\begin{definition}[Asymptotic Stability]
     The route choice adjustment process given by~\eqref{eqn:project-dynamical-system} is \textbf{asymptotically stable} if it is stable and, for any initial flow pattern $\mathbf{x}_0\in \Omega$ there exists some  equilibrium flow pattern $\mathbf{x}^*$ such that $\mathbf{x}_0(s) \to \mathbf{x}^*$ as $s\to \infty$.
\end{definition}

\begin{theorem}\label{thm:convergence}
    The route choice adjustment process is \textit{asymptotically stable} if the link cost $t$ is strictly monotone increasing in the link flow.
\end{theorem}

\begin{lemma}
	The path-flow feasibility conditions can be expressed in differential dynamic logic.
\end{lemma}

\begin{proof}
As before, let $\mathbf{q}=(q_1, \ldots, q_d)$ denote the non-negative demand between each origin-destination pair on the network and let $M$ and $D$ denote the trip-path indicence matrix and link-path indicence matrix respectively.
$\mathbf{q}$ is given as an input to the traffic assignment problem and $M$  and $D$ are derived from the network topology, so we may regard these as objects as fixed.

Let $\mathbf{F}=(F_1,\ldots, F_p)$ denote a path flow.
From the definition of path demand feasibility the vector product $M\mathbf{F}$ must equal the demand vector $\mathbf{q}$.
We may express this in first order logic as
\begin{align}
	\land_{i\in Q} & q_i = \sum_{j\in P} M_{ij}F_j
	\intertext{Moreover, each path flow must be non-negative.}
	\land_{j\in P} & 0\leq F_j
\end{align}

\end{proof}

\begin{lemma}
	The link-flow feasibility conditions can be expressed in differential dynamic logic.
\end{lemma}

\begin{proof}
	\begin{align}
		\Land_{i\in L} & (0\leq x_i)\\
		\exists F_1 \geq 0 \exists F_2 \geq 0 \cdots \exists F_p \geq 0 ( (\Land_{i\in Q} & q_i = \sum_{j\in P} M_{ij}F_j) \land (\Land_{k\in L}   x_k = \sum_{j \in P} D_{kj}F_j) )
	\end{align}
\end{proof}

\begin{remark}
    Although the constraint set is expressible in differential dynamic logic, the projected dynamic system is not.
    The projection operator in this setting is equivalent to projection onto a simplex. This is not generally attainable in closed form. However, as a special case, there exists a closed form for projection onto the 2-simplex which we leverage in the worked example in \cref{sec:single-player-example}.
\end{remark}

From \cref{thm:correspondence} we may prove that the equilibrium condition $P \equiv \forall \mathbf{\tilde{x}} \in \Omega (\mathbf{t}(\mathbf{x})(\mathbf{\tilde{x}}- \mathbf{x}) \geq 0)$ is an invariant of the route choice adjustment process~\eqref{eqn:project-dynamical-system}.
Moreover, \cref{thm:convergence} guarantees that following the route adjustment process will get us closer to equilibrium, or at least never lead us further from equilibrium.
Concretely, these two facts can be expressed in differential dynamic logic as follows.

\begin{theorem}
    The following is a valid formula.
    $$\forall \mathbf{\tilde{x}} \in \Omega (\mathbf{t}(\mathbf{x})(\mathbf{\tilde{x}}- \mathbf{x}) \geq 0) \implies [\mathbf{x}'=\Pi_{\Omega}(\mathbf{x}, -\mathbf{t}(\mathbf{x}))\ \&\ \mathbf{x}\in \Omega] \forall \mathbf{\tilde{x}} \in \Omega (\mathbf{t}(\mathbf{x})(\mathbf{\tilde{x}}- \mathbf{x}) \geq 0)$$
\end{theorem}

\begin{proof}
    By \eqref{eqn:ui-vi}, the pre-condition implies that $\mathbf{x}$ is an equilibrium point.
    By \cref{thm:correspondence}, we have that since $\mathbf{x}$ is an equilibrium point,  $\Pi_{\Omega}(\mathbf{x}, -\mathbf{t}(\mathbf{x})) = 0$.
    Therefore, $\mathbf{x}'=0$ initially. Because $\mathbf{x}'$ depends on $\mathbf{x}$ only, $\mathbf{x}$ will remain unchanged forever so the post-condition is true.
\end{proof}

\begin{theorem}
    \label{thm:dl:convergence}
    Let $\Omega^*$ denote the set of equilibria.
    If $\mathbf{t}: \Omega_s \to \mathbb{R}^m_+$ is strictly monotone increasing then the following is a valid formula.
    $$\mathbf{x} \in \Omega \implies \exists \mathbf{x}^*\in \Omega^* \forall \epsilon > 0\ \langle \mathbf{x}'=\Pi_{\Omega}(\mathbf{x}, -\mathbf{t}(\mathbf{x}))\ \&\ \mathbf{x}\in \Omega\rangle ( \|\mathbf{x} - \mathbf{x}^*\| \leq \epsilon)$$
\end{theorem}

\begin{proof}
    Let $\mathbf{x}_0 \in \Omega$. Let $\mathbf{x}_0(s)$ denote the state of $\mathbf{x}_0$ after running the dynamics for duration $s\geq 0$.
    Note that by construction, the dynamics ensure that if $\mathbf{x}$ is in $\Omega$ initially then $\mathbf{x}$ will remain in $\Omega$ after running the dynamics for any duration, i.e.\ the domain constraint is always satisfied.
    By \cref{thm:convergence}, for every $\epsilon > 0$ there exists real number $N$ such that if $s > N$ then $\|\mathbf{x}_0(s) - \mathbf{x}\| \leq \epsilon$. 
    Concretely, we may take any $r > N$.
    Then $\mathbf{x}_0(r) \in \Omega$ and $\mathbf{x}_0(s) \in \Omega$ for all $0 \leq s \leq r$.
    Moreover, we have that there exists some $\mathbf{x}^*\in \Omega^*$  such that $\|\mathbf{x}_0(r) - \mathbf{x}^*\| \leq \epsilon$.
\end{proof}

\subsection{Path swap dynamics}

\citet{smith1984stability} introduces a dynamical system formulation of traffic equilibrium in terms of path flow.
First, some notation. As above, $\mathbf{F}=(F_1, \cdots, F_p)$ is a vector of path flows.
Define the relation over paths $i\sim j$ as true if and only if $i$ and $j$ join the same origin destination pair.
Let $\mathbf{T}(\mathbf{F})=(T_1(\mathbf{F}), \cdots, T_p(\mathbf{F}))$ denote the path cost as a function of path flow.
Finally, let $(\cdot)_+=\max\{0, \cdot\}$.
We can rephrase Wardrop's equilibrium condition logically as $ i \sim j \land T_i(\mathbf{F}) > T_j(\mathbf{F}) \implies F_i = 0$ for all paths $i, j$.
Equivalently, $\mathbf{F}$ is an equilibrium path flow if and only if whenever $i\sim j$ \eqref{eqn:ue:smith-condition} holds.

\begin{align}
    \Phi_{ij}(\mathbf{F}) \equiv F_i(T_i(\mathbf{F}) - T_j(\mathbf{F}))_+ &= 0\label{eqn:ue:smith-condition}
\end{align}

We note that $\Phi_{ij}(\mathbf{F}) \geq 0$ for all demand feasible path flows because $F_i\geq 0$ for all $i$.
As a result, whenever $\mathbf{F}$ is not an equilibrium path flow $\Phi_{ij}(\mathbf{F}) > 0$.

Intuitively, $\Phi_{ij}$ can be thought of as the rate at which drivers change routes from route $i$ to route $j$.
Concretely suppose there are only two paths in the network, $i$ and $j$.
Further suppose $T_i(\mathbf{F}) > T_j(\mathbf{F})$.
Then drivers abandon route $i$ a rate of $\Phi_{ij}$ and clamor to route $j$ at a rate of $\Phi_{ij}$.
In other words, $F_i' = -\Phi_{ij}(\mathbf{F})$ and $F_j' = \Phi_{ij}(\mathbf{F})$.

To express the dynamical system in general we let $e_i\in \mathbb{R}^p$ denote the standard basis vector and define the ``swap'' vector $\Delta_{ij} = e_j - e_i$ if $i \sim j$ and $0$ otherwise.

\begin{align}
    \Delta T_{ij}(\mathbf{F}) &= -\mathbf{T}(\mathbf{F})\Delta_{ij} = T_i(\mathbf{F}) - T_j(\mathbf{F})\\
    \intertext{Importantly, this implies the identity $\Delta T_{ij} = -\Delta T_{ji}$. In particular, we can express $\Phi_{ij}$ as,}
    \Phi_{ij}(\mathbf{F}) &= F_i (\Delta T_{ij})_+\\
    \intertext{Finally, the dynamics are given by \eqref{eqn:ue:smith-dynamics}.}
    \mathbf{F}' &= \Phi(\mathbf{F}) \equiv \sum_{i, j\in P} \Phi_{ij}(\mathbf{F})\Delta_{ij} \label{eqn:ue:smith-dynamics}
\end{align}

The quantity $\Phi_{ij}$ can be thought of as the speed and $\Delta_{ij}$ the direction of $F$ under the influence of swapping higher cost routes for lower cost ones.

We briefly repeat the main results from \citet{smith1984stability}.

\begin{enumerate}
    \item Solutions to the differential equation $F'=0$ are equilibrium traffic flows. Related, equilibria are preserved as an invariant of the dynamics.
    \item Feasible non-equilibrium traffic flows converge to equilibrium traffic flows under path swap dynamics.
    \item Feasibility is preserved as an invariant under path swap dynamics.
\end{enumerate}

\begin{remark}
Importantly, the path swap dynamics in \eqref{eqn:ue:smith-condition} \textit{are} expressible in differential dynamic logic extended with the max function provided that the path flow function $T$ are expressible in differential dynamic logic.
\end{remark}

This fact follows transparently from \eqref{eqn:ue:smith-dynamics} which contains only additions, multiplications, and the max function. 

