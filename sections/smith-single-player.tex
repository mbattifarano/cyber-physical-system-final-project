
\subsection{Worked example}

Concretely, let us apply the dynamical systems formulation to the Braess network previously introduced.
First, let us compute the path cost functions from the link flow functions.
The path costs are simply the sum of the link costs along the path and the link flows are simply the sum of the path flows that utilize that link.
For clarity, the equations are first specified as a sum of link flows then simplified.
Here we use subscripts $u$, $l$, and $z$ for the upper, lower, ans zig-zag paths respectively. 

\begin{align}
    T_u(\mathbf{F}) &= 10(F_1+F_2) + 50 + F_1 = 50 + 10F_2 + 11F_1\\
    T_z(\mathbf{F}) &= 10(F_1+F_2) + 10 + F_2 + 10(F_2 + F_3) = 10 + 10 F_1 + 10F_3 21F_2 \\
    T_l(\mathbf{F}) &= 50 + F_3 + 10 (F_2 + F_3) = 50 + 10F_2 + 11F_3
\end{align}

Then we can write the dynamics according to \eqref{eqn:ue:smith-dynamics}.

\begin{align}
    \mathbf{F}' = \Phi(\mathbf{F}) = 
        \begin{pmatrix}
           F_u'\\
           F_z'\\
           F_l'
        \end{pmatrix} &=
        \begin{pmatrix}
            \Phi_{zu} - \Phi_{uz} + \Phi_{lu} - \Phi_{ul}\\
            \Phi_{uz} - \Phi_{zu} + \Phi_{lz} - \Phi_{zl}\\
            \Phi_{ul} - \Phi_{lu} + \Phi_{zl} - \Phi_{lz}\\
        \end{pmatrix}
\end{align}

Where we omit the dependence of $\Phi_{ij}$ on $\mathbf{F}$ for notational clarity.
We pause briefly to confirm out intuition about this system. 
The first differential, $F_u'$, represents the change in flow on the upper path over time.
The first half of the expression, $\Phi_{zu} - \Phi_{uz}$ represents the net movement of volume between the upper and zig zig paths and the second,  $\Phi_{zu} - \Phi_{uz}$ represents the net movement of volume between the upper and lower paths.
In total, the flow on the upper path changes by the volume leaving the zig-zig (lower) path for the upper path minus the volume leaving the upper path for the zig-zag (lower) path. 

We can concretely demonstrate a few desirable properties.

\begin{enumerate}
    \item $\Land_{i\in P} F_i \geq 0$ is an invariant of the dynamics.
        
        Proof by differential ghost separately for each $F_i$: $g' = \frac{1}{2}g \sum_{j\in P} (\Delta T_{ij})_+$.
    \item $\sum_{i\in P} F_i = q$ is an invariant of the dynamics.
    
        Proof by differential invariant.
    \item $\Phi(\mathbf{F}) = 0$ is an invariant of the dynamics.
        
        Proof by Remark in \citet{smith1984stability}. However, it is also relatively straight forward to show via differential invariant.
    \item $\Phi(\mathbf{F}) \to 0$ as the dynamics run.
    
        Proof by the stability theorem in \citet{smith1984stability}.
\end{enumerate}

The first two properties demonstrate that feasibility is invariant with respect to the dynamics.
The third states if the path flow is at equilibrium it remains at equilibrium for all time. Finally, the fourth states that the dynamics push a non-equilibrium state toward an equilibrium state.


\section{Single Player Dynamics}

Suppose a traffic controller is able to influence the link (path) cost functions, for example by applying a toll to one or more links.
Here we are interested in the equilibrium with respect to the the value of the control parameter $y\in Y \subseteq \mathbb{R}$.
For simplicity we assume that $Y$ is a closed interval in $\mathbb{R}$.
First, the path cost functions are now additionally dependent on $y$: $\mathbf{T}(y, \mathbf{F}) = (T_1(y,\mathbf{F}), \cdots, T_p(y, \mathbf{F})$.
Extending the notation from the previous section we have,

\begin{align}
    \Delta T_{ij}(y, \mathbf{F}) &= -\mathbf{T}(y, \mathbf{F})\Delta_{ij} = T_i(y, \mathbf{F}) - T_j(y, \mathbf{F})\\
    \Phi_{ij}(y, \mathbf{F}) &= F_i(\Delta T_{ij}(y, \mathbf{F}))_+\\
    \mathbf{F}' = \Phi(y, \mathbf{F}) &= \sum_{i,j \in P} \Phi_{ij}(y, \mathbf{F}) \Delta_{ij}
\end{align}

As before, a path flow $\mathbf{F}$ is an equilibrium path flow if and only if $\Phi(y, \mathbf{F}) = 0$.

For any fixed value of $y$ the dynamics are uninteresting.
All results from the previous section still hold.
We want to understand the behavior of the system as $y$ changes over the evolution time of the dynamical system.
There are two obvious ways to achieve this.
One option is to express the system as a non-adversarial differential game typically analyzed in the setting of optimal control theory \citep{bardi2008optimal}.
A second option is to express the changing value of $y$ as a differential, which is the approach we will take here.

In particular we are interested in the system given by \eqref{eqn:ue:single-player-diff:1}.

\begin{subequations}
    \begin{align}
        \mathbf{F}' &= \Phi(y, \mathbf{F})\\ 
        y' &= h(y, \mathbf{F})
    \end{align}\label{eqn:ue:single-player-diff:1}
\end{subequations}

subject to the evolution domain constrain that $y\in Y$.

The first observation is that equilibria are not, in general, invariants of \eqref{eqn:ue:single-player-diff:1}.
It is clear from the dependence on $y$ in $\Phi$ that any for any non-trivial choice of $h$ and $\mathbf{C}(\cdot, \mathbf{F})$, $\Phi(y, \mathbf{F})$ will stray from 0 as soon as the dynamics start to run.

We would like instead to define a new function $\hat{\Phi}(y, \mathbf{F})$ which compensates for how $y$ is changing in order for $\Phi(y, \mathbf{F}) = 0$ to be an invariant. Concretely,

\begin{subequations}
    \begin{align}
        \mathbf{F}' &= \hat{\Phi}(y, \mathbf{F})\\ 
        y' &= h(y, \mathbf{F})
    \end{align}\label{eqn:ue:single-player-diff:2}
\end{subequations}