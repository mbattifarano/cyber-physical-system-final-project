\section{Single-Player Control}

Suppose a traffic controller is able to influence the link cost function, for example, by applying a toll to one or more links.
Here we are interested in examining the dynamical system governed by \eqref{eqn:single-player-dynamics}. 

\begin{align}
    \mathbf{x}' &= \Pi_{\Omega} (\mathbf{x}, -\mathbf{t}(\mathbf{y}, \mathbf{x}))\label{eqn:single-player-dynamics}
\end{align}

Where $\mathbf{y}$ is typically a function of the evolution ``time'' of the system.
This is a trivial differential game in the sense that the dynamics do not additionally depend on a second players control.
It is typically analyzed in the setting of optimal control theory. \citep{bardi2008optimal}

In this section appropriate conditions on $\mathbf{t}$ and the control input $\mathbf{y}$ will be developed so that \crefrange{thm:correspondence}{thm:convergence} hold.

To aid in our discussion, we say that traffic flow $\mathbf{x}$ is at equilibrium with respect to control parameters $\mathbf{y}$ if $\Pi_{\Omega} (\mathbf{x}, -\mathbf{t}(\mathbf{y}, \mathbf{x}))=0$.

\subsection{Discrete case}

In the discrete case, we are interested in showing that the dynamical system will converge to equilibrium after a discrete change to the control parameters.

\begin{theorem}[Discrete control]
Let $Y$ denote domain of control parameters and $\Omega^*_y$ denote the set of equilibria with respect to $y$. Let $\mathbf{t}: Y\times \Omega_s \to \mathbb{R}^m_+$ be strictly monotone increasing in $\mathbf{x}$ for any fixed $y\in Y$. The following is a valid formula.

$$\mathbf{x} \in \Omega \implies \forall \epsilon > 0\  \mathbf{y}\assigned *;?(\mathbf{y}\in Y); \langle \mathbf{x}'=\Pi_{\Omega}(\mathbf{x}, -\mathbf{t}(\mathbf{y}, \mathbf{x}))\ \&\ \mathbf{x}\in \Omega\rangle (\exists \mathbf{x}^*\in \Omega^*_\mathbf{y}\ \|\mathbf{x} - \mathbf{x}^*\| \leq \epsilon)$$
\end{theorem}

\begin{proof}
We proceed with the proof using sequent calculus.

\begin{prooftree}
    \AxiomC{*}
    \LeftLabel{\Cref{thm:dl:convergence}}
    \UnaryInfC{$\mathbf{x} \in \Omega, \mathbf{y}\in Y \vdash \forall \epsilon > 0\ \langle \mathbf{x}'=\Pi_{\Omega}(\mathbf{x}, -\mathbf{t}(\mathbf{y}, \mathbf{x}))\ \&\ \mathbf{x}\in \Omega\rangle (\exists \mathbf{x}^*\in \Omega^*_\mathbf{y}\ \|\mathbf{x} - \mathbf{x}^*\| \leq \epsilon)$}
    \UnaryInfC{$\mathbf{x} \in \Omega, \epsilon > 0, \mathbf{y}\in Y \vdash \langle \mathbf{x}'=\Pi_{\Omega}(\mathbf{x}, -\mathbf{t}(\mathbf{y}, \mathbf{x}))\ \&\ \mathbf{x}\in \Omega\rangle (\exists \mathbf{x}^*\in \Omega^*_\mathbf{y}\ \|\mathbf{x} - \mathbf{x}^*\| \leq \epsilon)$}
    \UnaryInfC{$\mathbf{x} \in \Omega \implies \forall \epsilon > 0\  \mathbf{y}\assigned *;?(\mathbf{y}\in Y); \langle \mathbf{x}'=\Pi_{\Omega}(\mathbf{x}, -\mathbf{t}(\mathbf{y}, \mathbf{x}))\ \&\ \mathbf{x}\in \Omega\rangle (\exists \mathbf{x}^*\in \Omega^*_\mathbf{y}\ \|\mathbf{x} - \mathbf{x}^*\| \leq \epsilon)$}
\end{prooftree}

We may apply \cref{thm:dl:convergence} in the last step because $\mathbf{y}$ is fixed in the differential equation.
As a result, $\mathbf{t}(\mathbf{y}, \mathbf{x}))$ may be re-written as a function of $\mathbf{x}$ alone, say $\mathbf{t}_y(\mathbf{x})$, which meets the precondition of \cref{thm:dl:convergence}.
\end{proof}

What if the system starts off at equilibrium?
In this case we would like to show that a small change in control parameters results in a small change in equilibrium toward which the dynamics will evolve.

\begin{theorem}
\label{thm:di:eq-continuity}
Let $Y$ denote domain of control parameters and $\Omega^*_y$ denote the set of equilibria with respect to $y$.
Let $\mathbf{t}: Y\times \Omega_s \to \mathbb{R}^m_+$ be continuous and strictly monotone increasing in $\mathbf{x}$ for any fixed $y\in Y$.

The following are valid formulae.

\begin{align*}
    \mathbf{x} \in \Omega^*_\mathbf{y} &\implies \forall \epsilon > 0\ \exists \delta >0\ \mathbf{y}\assigned \mathbf{y} + \delta;?(\mathbf{y}\in Y); [\mathbf{x}'=\Pi_{\Omega}(\mathbf{x}, -\mathbf{t}(\mathbf{y}, \mathbf{x}))\ \&\ \mathbf{x}\in \Omega] (\exists \mathbf{x}^*\in \Omega^*_\mathbf{y}\ \|\mathbf{x} - \mathbf{x}^*\| \leq \epsilon)\\
    \mathbf{x} \in \Omega^*_\mathbf{y} &\implies \forall \epsilon > 0\ \forall \delta >0\ \mathbf{y}\assigned \mathbf{y} + \delta;?(\mathbf{y}\in Y); \langle\mathbf{x}'=\Pi_{\Omega}(\mathbf{x}, -\mathbf{t}(\mathbf{y}, \mathbf{x}))\ \&\ \mathbf{x}\in \Omega\rangle (\exists \mathbf{x}^*\in \Omega^*_\mathbf{y}\ \|\mathbf{x} - \mathbf{x}^*\| \leq \epsilon)
\end{align*}

\end{theorem}

\subsection{Continuous control}

Consider the case in which $\mathbf{t}$ is continuous in both the traffic flow $\mathbf{x}$ and control parameter $\mathbf{y}$.
It would be nice if the continuity of $\mathbf{t}$ extended to some notion of continuity of the equilibrium state.
In particular, suppose $\mathbf{x}$ is an equilibrium traffic flow with respect to $\mathbf{y}$.
We then suppose that we follow the system of differential equations given by \eqref{eqn:single-player-diff-sys} for some $r\geq 0$.

\begin{subequations}
\begin{align}
    \mathbf{x}' &= \Pi_{\Omega} (\mathbf{x}, -\mathbf{t}(\mathbf{y}, \mathbf{x}))\\
    \mathbf{y}' &= f(\mathbf{y})
\end{align}
\label{eqn:single-player-diff-sys}
\end{subequations}

Where $f$ is some function definable in first order arithmetic.
This is a slight re-phrasing of 
Denote the value of $\mathbf{x}$ and $\mathbf{y}$ after running the system for duration $r$ by $\mathbf{x}^{(r)}$ and $\mathbf{y}^{(r)}$.
One reasonable question arises: if $\mathbf{x}$ was an equilibrium with respect to $\mathbf{y}$ is $\mathbf{x}^{(r)}$ an equilibrium with respect to $\mathbf{y}^{(r)}$?

Intuitively, we ought to be able to find conditions under which $\mathbf{x}^{(r)}$ is ``close'' to the equilibrium with respect to $\mathbf{y}^{(r)}$.
Informally, we can think of this as a continuous version of the model in \cref{thm:di:eq-continuity}, again starting from an equilibrium traffic flow $x$.
\begin{enumerate}
    \item the control parameter $y$ is changed by some small about $\delta$
    \item the traffic flow is no longer at equilibrium with respect to $y$, but by following the dynamics the link flow approaches the equilibrium with respect to (the updated) control parameter.
\end{enumerate}
In a loop, these two steps can be thought of as a chase of some kind: a discrete change is made to the control and then the traffic flow ``catches up`` to the new equilibrium.
We would rightly worry that over many iterations the traffic state would stray arbitrarily far from the equilibrium unless the dynamics were allowed to run long enough.
However, this scenario has a natural expression as a hybrid game in which one player (the traffic controller) sets $y$ and the number of iterations, while the other (the traffic) determines how long to run the differential.

In the continuous control system \eqref{eqn:single-player-diff-sys} both the control parameters \textit{and} the link flows are changing smoothly and simultaneously. 
We would therefore like to establish three facts:
\begin{enumerate}
    \item The link flow derivative is \textit{always} pointed toward the equilibrium with respect to the current value of $y$.
    \item We may bound $f$ in such a way to ensure that if $x$ is initially an equilibrium, then the current value of $x$ will always remain close to equilibrium with respect to the current value for $y$.
    \item We may bound $f$ in such a way to ensure that if $x$ is initially close to equilibrium with respect to $y$, then  it will always remain close to equilibrium with respect to $y$.
    \item Are there any conditions on $f$ or $\mathbf{y}$ which cause the system to remain at equilibrium with respect to $\mathbf{y}$ as an invariant? I don't think so. Would this be possible if we added a second (adversarial) player who compensates for the changes to the control parameter?
    \item Finally, is there an natural expression of the continuous system as a differential game?
\end{enumerate}