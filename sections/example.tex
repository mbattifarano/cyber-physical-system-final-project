\section{A motivating example}\label{sec:motivating-example}

Consider the morning commute.
Suppose every morning a set of drivers each leave their homes at the same time each headed to their respective workplaces.
Every morning each driver decides which path they should take to work.
They do not know what paths any of the other drivers will choose today, but each knows what happened yesterday.
In particular, we suppose that each has full information of what happened yesterday. 

Consider the following behavioral principle: drivers only change route in such a way to reduce travel time \textit{based on yesterday's travel costs}.
We suppose that drivers follow this principle when deciding which route to take every morning.
Note that they need not change route from day to day, but if they do the chosen route \textit{must} decrease travel costs.
The fixed points of this decision principle are those route choices which \textit{cannot} change from one day to the next.
According to the decision principle yesterday's route choices cannot change if and only if no driver has a less costly alternative route.
This coincides exactly with Wardrop's first condition so the fixed points of this decision processes are exactly the equilibria as defined by Wardrop.

We will return to this example in subsequent sections to motivate several concepts.